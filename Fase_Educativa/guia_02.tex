\documentclass[12pt,letterpaper]{article}
% -------------------------------------------------------------------
% PAQUETES BÁSICOS
% -------------------------------------------------------------------
\usepackage[utf8]{inputenc}
\usepackage[T1]{fontenc}
\usepackage[spanish]{babel}
\usepackage{lmodern}
\usepackage{setspace}
\onehalfspacing

% -------------------------------------------------------------------
% MÁRGENES Y ESTILO DE DOCUMENTO
% -------------------------------------------------------------------
\usepackage[margin=2.5cm]{geometry}
\usepackage{parskip}
\setlength{\parindent}{0pt}

% -------------------------------------------------------------------
% PAQUETES PARA CÓDIGO (MINTED)
% REQUIERE COMPILAR CON:   pdflatex -shell-escape archivo.tex
% -------------------------------------------------------------------
\usepackage{minted}
\newminted{shell-session}{
    fontsize=\small,
    breaklines=true,
    autogobble,
    frame=single,
    framesep=3mm
    
}
\setminted{
    fontsize=\small,
    breaklines=true,
    linenos=false,
    autogobble,
    frame=single,
    framesep=3mm,
    breakanywhere=true
}

% -------------------------------------------------------------------
% PAQUETES ADICIONALES
% -------------------------------------------------------------------
\usepackage{hyperref}
\hypersetup{
    colorlinks=true,
    linkcolor=blue,
    urlcolor=blue
}

\usepackage{tcolorbox}
\tcbuselibrary{listings,skins,breakable}
% CAJAS DE ADVERTENCIA / NOTA
\tcbset{colback=gray!10, colframe=gray!70, sharp corners}

\newenvironment{nota}{
    \begin{tcolorbox}[title=Nota]
}{
    \end{tcolorbox}
}

\newenvironment{advertencia}{
    \begin{tcolorbox}[colback=red!10,colframe=red!60,title=Advertencia]
}{
    \end{tcolorbox}
}
\usepackage{fancyhdr}
\setlength{\headheight}{15pt}
% -------------------------------------------------------------------
% DOCUMENTO
% -------------------------------------------------------------------
\usepackage{graphicx} % IMPORTANTE
\usepackage{background}
\usepackage{float}

\backgroundsetup{
  scale=0.8,
  angle=0,
  opacity=0.1,
  contents={\includegraphics[width=\paperwidth]{images/firma-vertical-dos-lineas-negro.png}}
}

% Set the page style to "fancy"...
\pagestyle{fancy}
%... then configure it.
\fancyhead{} % clear all header fields
\fancyhead[L]{\textbf{Guía de uso del Simulador Fluidsim}}
\fancyfoot{} % clear all footer fields
\fancyfoot[R]{\thepage}
\fancyfoot[L]{Gustavo Barboza Blanco - UCR}

\begin{document}

\section{Introduión a Fluidsim}

Fluidsim es una librería científica escrita en \textbf{Python}, diseñada
para la simulación numérica de flujos mediante métodos
\textbf{pseudoespectrales}. Está pensada principalmente para investigación
y docencia en dinámica de fluidos.

El simulador no es una aplicación con interfaz gráfica,
sino un \textbf{framework de simulación}, en el cual el usted define
un conjunto de parámetros físicos y numéricos, y luego ejecuta la
simulación mediante scripts en Python.

\section{Filosofía y arquitectura del simulador}

Fluidsim sigue una arquitectura modular y orientada a objetos.
Cada simulación se construye a partir de los siguientes componentes:

\begin{itemize}
    \item Un \textbf{solver}, que define las ecuaciones físicas a resolver.
    \item Un conjunto de \textbf{parámetros}, que controlan el dominio,
    la discretización, la integración temporal y la física del sistema.
    \item Un sistema de \textbf{tiempo}, encargado de la integración temporal.
    \item Un sistema de \textbf{salida}, que gestiona los archivos de resultados.
\end{itemize}

El usuario interactúa principalmente con el solver y con los parámetros,
mientras que el resto del flujo es gestionado internamente por Fluidsim.

\section{Concepto de solver en Fluidsim}

En Fluidsim, un solver representa un \textbf{modelo físico específico}.
Por ejemplo, existen solvers para:

\begin{itemize}
    \item Navier–Stokes en 2D
    \item Navier–Stokes en 3D
    \item Turbulencia cuasi-geostrófica
\end{itemize}

Cada solver define: las ecuaciones diferenciales a resolver,
el método numérico utilizado y las variables dinámicas del sistema.

Puede ver todos los \texttt{solvers} disponibles en la \href{https://fluidsim.readthedocs.io/en/latest/generated/fluidsim.solvers.html}{documentación},
ahí mismo podrá ver los parámetros que componen cada uno. 


\section{Parámetros de simulación}

Los parámetros de una simulación se almacenan en un objeto jerárquico,
el cual contiene tanto parámetros físicos como numéricos.

Estos parámetros se crean mediante el método:

\begin{minted}{python}
from fluidsim.solvers.ns2d.solver import Simul
params = Simul.create_default_params()
\end{minted}

Este objeto contiene información sobre:
\begin{itemize}
    \item El tamaño del dominio espacial.
    \item La resolución numérica.
    \item El paso de tiempo.
    \item Las constantes físicas del modelo.
\end{itemize}


\section{Ciclo de vida de una simulación}

Una simulación típica en Fluidsim sigue siempre el mismo flujo:

\begin{enumerate}
    \item Creación del objeto de parámetros.
    \item Inicialización del solver.
    \item Integración temporal.
    \item Escritura de resultados.
\end{enumerate}

Es decir:

\begin{minted}{python}
#Creación del objeto
sim = Simul(params)
#Se inicia la simulación
sim.time_stepping.start()
\end{minted}

Al ejecutar \texttt{start()}, Fluidsim se encarga automáticamente
de iterar en el tiempo, aplicar los operadores espectrales y guardar
los resultados según la configuración.


\section{Ejemplo mínimo completo}

En esta sección se explicará un ejemplo básico de forma progresiva. 
La idea es dar a entender cómo se hacen las simulaciones con esta herramienta,
en general el algoritmo a seguir es el mismo, y consiste principalmente 
en cómo manipular los parámetros de un solver. Por suerte el resto
del trabajo lo maneja el Simulador. 

Primero se elige el modelo físico con el que vamos a trabajar, digamos que 
en este caso nos interesa trabajar con Navier-Stokes2D para un fluido
incompresible. Para ello debemos importar \texttt{Simul}, que es el objeto principal
de una simulación:

\begin{minted}{python}
    from fluidsim.solvers.ns2d.solver import Simul
\end{minted}

Ahora viene una de las partes más importantes, configurar los parámetros
de nuestra simulación. Aquí podemos modificar elementos como: Dominio espacial,
Resolución, integración temporal y parámetros físicos como la viscocidad 
y forzamientos:

\begin{minted}{python}
    #Creamos el objeto params
    params = Simul.create_default_params()
    #Ahora podemos modificar cosas:
    
    #Puntos de discretización
    params.oper.nx = 64
    params.oper.ny = 64

    #Tiempo final de la simulación (Empieza en t=0)
    params.time_stepping.t_end = 5.0

\end{minted}

Es importante recalcar todo el trabajo que \texttt{Fluidsim}
hace por debajo, en este caso el paso del tiempo lo calcula el simulador 
según la estabilidad numérica.

Ahora le damos vida a la simulación como objeto:

\begin{minted}{python}
    sim = Simul(params)
\end{minted}

Aquí suceden muchas cosas de forma interna: se construye la malla espectral,
se inicializan campos, se configuran operadores FFT,
se crean los manejadores de salida. 

Finalmente: 

\begin{minted}{python}
    sim.time_stepping.start()
\end{minted}

Aquí se inicia el bucle de integración temporal. Fluidsim ahora: avanza en el tiempo,
resuelve las ecuaciones, guarda los datos e imprime información en pantalla.

Conceptualmente, el simulador ``cobra vida''.

El código completo se verá así:
\begin{minted}{python}
from fluidsim.solvers.ns2d.solver import Simul

params = Simul.create_default_params()

params.oper.nx = 64
params.oper.ny = 64
params.time_stepping.t_end = 5.0

sim = Simul(params)
sim.time_stepping.start()
\end{minted}

Con este entendimiento de la estructura básica de una simulación, se le 
recomienda al usuario, recrear los tutoriales y ejemplos que presenta
la \href{https://fluidsim.readthedocs.io/en/latest/tutorials.html}{documentación oficial}.

\section{Ejecución, salida y post-procesamiento}

Fluidsim guarda los resultados automáticamente en un directorio de salida,
el cual contiene archivos con los campos simulados.

Estos archivos pueden analizarse posteriormente mediante scripts en Python
o herramientas como Jupyter Notebook.

\end{document}