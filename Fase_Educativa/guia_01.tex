\documentclass[12pt,letterpaper]{article}
% -------------------------------------------------------------------
% PAQUETES BÁSICOS
% -------------------------------------------------------------------
\usepackage[utf8]{inputenc}
\usepackage[T1]{fontenc}
\usepackage[spanish]{babel}
\usepackage{lmodern}
\usepackage{setspace}
\onehalfspacing

% -------------------------------------------------------------------
% MÁRGENES Y ESTILO DE DOCUMENTO
% -------------------------------------------------------------------
\usepackage[margin=2.5cm]{geometry}
\usepackage{parskip}
\setlength{\parindent}{0pt}

% -------------------------------------------------------------------
% PAQUETES PARA CÓDIGO (MINTED)
% REQUIERE COMPILAR CON:   pdflatex -shell-escape archivo.tex
% -------------------------------------------------------------------
\usepackage{minted}
\newminted{shell-session}{
    fontsize=\small,
    breaklines=true,
    autogobble,
    frame=single,
    framesep=3mm
    
}
\setminted{
    fontsize=\small,
    breaklines=true,
    linenos=false,
    autogobble,
    frame=single,
    framesep=3mm,
    breakanywhere=true
}

% -------------------------------------------------------------------
% PAQUETES ADICIONALES
% -------------------------------------------------------------------
\usepackage{hyperref}
\hypersetup{
    colorlinks=true,
    linkcolor=blue,
    urlcolor=blue
}

\usepackage{tcolorbox}
\tcbuselibrary{listings,skins,breakable}
% CAJAS DE ADVERTENCIA / NOTA
\tcbset{colback=gray!10, colframe=gray!70, sharp corners}

\newenvironment{nota}{
    \begin{tcolorbox}[title=Nota]
}{
    \end{tcolorbox}
}

\newenvironment{advertencia}{
    \begin{tcolorbox}[colback=red!10,colframe=red!60,title=Advertencia]
}{
    \end{tcolorbox}
}
\usepackage{fancyhdr}
\setlength{\headheight}{15pt}
% -------------------------------------------------------------------
% DOCUMENTO
% -------------------------------------------------------------------
\usepackage{graphicx} % IMPORTANTE
\usepackage{background}
\usepackage{float}

\backgroundsetup{
  scale=0.8,
  angle=0,
  opacity=0.1,
  contents={\includegraphics[width=\paperwidth]{images/firma-vertical-dos-lineas-negro.png}}
}

% Set the page style to "fancy"...
\pagestyle{fancy}
%... then configure it.
\fancyhead{} % clear all header fields
\fancyhead[L]{\textbf{Guía de uso del Simulador Fluidsim}}
\fancyfoot{} % clear all footer fields
\fancyfoot[R]{\thepage}
\fancyfoot[L]{Gustavo Barboza Blanco - UCR}

\begin{document}
\section{Introducción}

En esta guía usted aprenderá a instalar el simulador 
\href{https://fluidsim.readthedocs.io/en/latest/index.html}{Fluidsim}
que forma parte del ecosistema 
\href{https://fluiddyn.readthedocs.io/en/latest/}{FluidDyn} 
y funciona con \textbf{Python}. La instalación se 
realizará utilizando el entorno \texttt{mamba} por medio de la distribución  
\texttt{Miniforge3}. 


Fluidsim recomienda explícitamente Miniforge o Mambaforge, en lugar de Anaconda o Miniconda.
Esto evita conflictos con paquetes propietarios y resuelve mejor dependencias.

\section{Requisitos Previos}
Es necesario cumplir con los siguientes requisitos:

\begin{itemize}
    \item Conexión a internet.
    \item Sistema operativo Linux, Windows o macOS.
    \item Espacio libre en disco (mínimo 2 GB).
\end{itemize}

Si bien es posible utilizar el simulador en Windows, es más recomendable
utilizar Linux. Pues las utilidades necesarias no
funcionan de forma nativa en Windows, es necesario
instalar \texttt{WSL2}, lo cual funciona como una capa de compatibilidad
para entornos nativos de Linux dentro de Windows.

Otra alternativa viable es utilizar una máquina virtual.

\section{Instalación de Miniforge3}
Primero debe abrir la terminal, verifique que está en la dirección
\fbox{\textbackslash home} del sistema operativo. Si está en
\texttt{Windows} debe usar \texttt{Powershell}.

Para obtener el instalador de \texttt{Miniforge3} ejecute el siguiente
comando en la terminal:

\begin{minted}{bash}
    wget https://github.com/conda-forge/miniforge/releases/latest/download/Miniforge3-Linux-x86_64.sh
\end{minted}

\textbf{Nota:} A la hora de copiar el código anterior, asegurese de que quede en una sola línea de comando.

Una vez obtenido el instalador, lo ejecutamos de la siguiente forma:

\begin{minted}{bash}
    bash Miniforge3-Linux-x86_64.sh
\end{minted}

Cierre la terminal y abrala de nuevo. En frente de la línea de comandos
deberá de leerse un \fbox{(base)} previo al directorio. 

\section{Instalación de Fluidsim}

Ahora ya se puede utilizar la herramienta \texttt{mamba}, para crear
entornos y manejar paquetes. El primer paso será crear un entorno o 
ambiente en donde se instalará todo lo referente al simulador. Ejecute 
la siguiente línea: 

\begin{minted}{bash}
    mamba create -n fluidsim python
\end{minted}

Las descargas pueden tardar un poco. 

Seguidamente debemos activar el entorno:

\begin{minted}{bash}
    mamba activate fluidsim
\end{minted}

Aquí verá que el entorno ya no es \fbox{(base)}, en su lugar será: \fbox{(fluidsim)}.

Al hacer esto hemos creado un entorno con su ``propio'' \textbf{Python} y \textbf{PATH},
todo esto se hace con el fin de no mezclar dependencias y evitar cualquier
tipo de problemas o inconsistencias dentro del sistema. 

Tal y como se recomienda en la documentación del simulador, se realizará
la instalación principal mediante el comando \textbf{pip}:

\begin{minted}{bash}
    pip install fluidsim
\end{minted}

Y también será necesario instalar la dependencia principal de algunos
solvers que utilizan el método pseudoespectral, junto la herramienta
para ejecutar simulaciones en paralelo con \texttt{MPI}

\begin{minted}{bash}
    pip install "fluidsim[fft,mpi]"
\end{minted}

Con esto se instala la versión base del simulador, sin embargo se 
recomiendan ciertas herramientas necesarias y útiles en el flujo
de trabajo usual dentro del simulador, para ello usaremos \texttt{mamba}.

Ejecute la siguiente línea:

\begin{minted}{bash}
    mamba install ipython numpy matplotlib ipympl scipy pandas jupyterlab jupyterlab-myst jupyterlab-spellchecker jupyterlab-variableinspector nb_conda_kernels ffmpeg
\end{minted}

\textbf{Nota:} Recuerde, a la hora de copiar el código anterior, asegurese de que quede en una sola línea de comando. 


\section{Prueba de instalación}

Para comprobar que la instalación se hizo de forma correcta, puede
ejecutar una simulación rápida en ipython, ejecute el siguiente comando
en la terminal:

\begin{minted}{bash}
ipython    
\end{minted}
Esto abrirá un ipython notebook, en donde basta con copiar y pegar el
siguiente código:
\begin{minted}{python}
    from fluidsim.solvers.ns2d.solver import Simul
    params = Simul.create_default_params()
    # Modify parameters as needed
    sim = Simul(params)
    sim.time_stepping.start()
\end{minted}

Luego presione enter y en la terminal deberá de ver un output como este:

\begin{figure}[H]
\includegraphics[width=12cm]{images/testOutput.png}
\centering
\caption{Ouptut esperado de la simulación básica}
\end{figure}

Esto significa que se logró ejecutar una simulación básica. Todo funciona
correctamente. 

\section{Flujo de trabajo típico en Fluidsim}

Recuerde que cada vez que quiera trabajar con el simulador debe de
activar el entorno correspondiente, en este caso llamado: \texttt{fluidsim}. 
Y una vez terminada la sesión de trabajo utilizando el simulador debe de desactivarlo
para volver a la versión base de su sistema. Para ello debe usar el comando:

\begin{minted}{bash}
    mamba deactivate 
\end{minted}

\begin{figure}[H]
\includegraphics[width=12cm]{images/workflow.pdf}
\centering
\caption{Flujo de trabajo típico en un entorno de mamba o conda}
\end{figure}

\subsection{Comentarios adicionales}
\begin{itemize}
    \item No es necesario trabajar en un notebook específico, usted puede
    realizar scripts en su editor de código preferido y luego ejecutarlos en 
    la terminal con el comando: \fbox{Python3 script.py}
    \item Si debe eliminar el entorno por alguna razón, debe de usar el 
    
    comando: \fbox{mamba env remove -n fluidsim}
\end{itemize}


\end{document}
