% -------------------------------------------------------------------
% PAQUETES BÁSICOS
% -------------------------------------------------------------------
\usepackage[utf8]{inputenc}
\usepackage[T1]{fontenc}
\usepackage[spanish]{babel}
\usepackage{lmodern}
\usepackage{setspace}
\onehalfspacing

% -------------------------------------------------------------------
% MÁRGENES Y ESTILO DE DOCUMENTO
% -------------------------------------------------------------------
\usepackage[margin=2.5cm]{geometry}
\usepackage{parskip}
\setlength{\parindent}{0pt}

% -------------------------------------------------------------------
% PAQUETES PARA CÓDIGO (MINTED)
% REQUIERE COMPILAR CON:   pdflatex -shell-escape archivo.tex
% -------------------------------------------------------------------
\usepackage{minted}
\newminted{shell-session}{
    fontsize=\small,
    breaklines=true,
    autogobble,
    frame=single,
    framesep=3mm
    
}
\setminted{
    fontsize=\small,
    breaklines=true,
    linenos=false,
    autogobble,
    frame=single,
    framesep=3mm,
    breakanywhere=true
}

% -------------------------------------------------------------------
% PAQUETES ADICIONALES
% -------------------------------------------------------------------
\usepackage{hyperref}
\hypersetup{
    colorlinks=true,
    linkcolor=blue,
    urlcolor=blue
}

\usepackage{tcolorbox}
\tcbuselibrary{listings,skins,breakable}
% CAJAS DE ADVERTENCIA / NOTA
\tcbset{colback=gray!10, colframe=gray!70, sharp corners}

\newenvironment{nota}{
    \begin{tcolorbox}[title=Nota]
}{
    \end{tcolorbox}
}

\newenvironment{advertencia}{
    \begin{tcolorbox}[colback=red!10,colframe=red!60,title=Advertencia]
}{
    \end{tcolorbox}
}
\usepackage{fancyhdr}
\setlength{\headheight}{15pt}
% -------------------------------------------------------------------
% DOCUMENTO
% -------------------------------------------------------------------
\usepackage{graphicx} % IMPORTANTE
\usepackage{background}
\usepackage{float}

\backgroundsetup{
  scale=0.8,
  angle=0,
  opacity=0.1,
  contents={\includegraphics[width=\paperwidth]{images/firma-vertical-dos-lineas-negro.png}}
}

% Set the page style to "fancy"...
\pagestyle{fancy}
%... then configure it.
\fancyhead{} % clear all header fields
\fancyhead[L]{\textbf{Guía de uso del Simulador Fluidsim}}
\fancyfoot{} % clear all footer fields
\fancyfoot[R]{\thepage}
\fancyfoot[L]{Gustavo Barboza Blanco - UCR}